\documentclass[catalysts,article,submit,moreauthors,pdftex,10pt,a4paper]{mdpi} 
\usepackage{todonotes}
\usepackage{comment}
\usepackage{graphicx}
%--------------------
% Class Options:
%--------------------
% journal
%----------
% Choose between the following MDPI journals:
% actuators, admsci, aerospace, agriculture, agronomy, algorithms, animals, antibiotics, antibodies, antioxidants, applsci, arts, atmosphere, atoms, axioms, batteries, behavsci, beverages, bioengineering, biology, biomedicines, biomimetics, biomolecules, biosensors, brainsci, buildings, carbon, cancers, catalysts, cells, challenges, chemosensors, children, chromatography, climate, coatings, computation, computers, condensedmatter, cosmetics, cryptography, crystals, data, dentistry, designs, diagnostics, diseases, diversity, econometrics, economies, education, electronics, energies, entropy, environments, epigenomes, fermentation, fibers, fishes, fluids, foods, forests, futureinternet, galaxies, games, gels, genealogy, genes, geosciences, geriatrics, healthcare, horticulturae, humanities, hydrology, informatics, information, infrastructures, inorganics, insects, instruments, ijerph, ijfs, ijms, ijgi, inventions, jcdd, jcm, jdb, jfb, jfmk, jimaging, jof, jintelligence, jlpea, jmse, jpm, jrfm, jsan, land, languages, laws, life, literature, lubricants, machines, magnetochemistry, marinedrugs, materials, mathematics, mca, mti, medsci, medicines, membranes, metabolites, metals, microarrays, micromachines, microorganisms, minerals, molbank, molecules, mps, nanomaterials, ncrna, neonatalscreening, nutrients, particles, pathogens, pharmaceuticals, pharmaceutics, pharmacy, philosophies, photonics, plants, polymers, processes, proteomes, publications, recycling, religions, remotesensing, resources, risks, robotics, safety, sensors, separations, sexes, sinusitis, socsci, societies, soils, sports, standards, sustainability, symmetry, systems, technologies, toxics, toxins, universe, urbansci, vaccines, vetsci, viruses, water
%---------
% article
%---------
% The default type of manuscript is article, but can be replaced by: 
% addendum, article, book, bookreview, briefreport, casereport, changes, comment, commentary, communication, conceptpaper, correction, conferencereport, expressionofconcern, meetingreport, creative, datadescriptor, discussion, editorial, essay, erratum, hypothesis, interestingimage, letter, newbookreceived, opinion, obituary, projectreport, reply, retraction, review, sciprints, shortnote, supfile, technicalnote
% supfile = supplementary materials
%----------
% submit
%----------
% The class option "submit" will be changed to "accept" by the Editorial Office when the paper is accepted. This will only make changes to the frontpage (e.g. the logo of the journal will get visible), the headings, and the copyright information. Also, line numbering will be removed. Journal info and pagination for accepted papers will also be assigned by the Editorial Office.
%------------------
% moreauthors
%------------------
% If there is only one author the class option oneauthor should be used. Otherwise use the class option moreauthors.
%---------
% pdftex
%---------
% The option pdftex is for use with pdfLaTeX. If eps figure are used, remove the option pdftex and use LaTeX and dvi2pdf.

%=================================================================
\firstpage{1} 
\makeatletter 
\setcounter{page}{\@firstpage} 
\makeatother 
\articlenumber{x}
\doinum{10.3390/------}
\pubvolume{xx}
\pubyear{2016}
\copyrightyear{2016}
\externaleditor{Academic Editor: name}
\history{Received: date; Accepted: date; Published: date}
%------------------------------------------------------------------
% The following line should be uncommented if the LaTeX file is uploaded to arXiv.org
%\pdfoutput=1

%=================================================================
% Add packages and commands here. The following packages are loaded in our class file: fontenc, calc, indentfirst, fancyhdr, graphicx, lastpage, ifthen, lineno, float, amsmath, setspace, enumitem, mathpazo, booktabs, titlesec, etoolbox, amsthm, hyphenat, natbib, hyperref, footmisc, geometry, caption, url, mdframed

%=================================================================
%% Please use the following mathematics environments:
 \theoremstyle{mdpi}
 \newcounter{thm}
 \setcounter{thm}{0}
 \newcounter{ex}
 \setcounter{ex}{0}
 \newcounter{re}
 \setcounter{re}{0}

 \newtheorem{Theorem}[thm]{Theorem}
 \newtheorem{Lemma}[thm]{Lemma}
 \newtheorem{Corollary}[thm]{Corollary}
 \newtheorem{Proposition}[thm]{Proposition}

 \theoremstyle{mdpidefinition}
 \newtheorem{Characterization}[thm]{Characterization}
 \newtheorem{Property}[thm]{Property}
 \newtheorem{Problem}[thm]{Problem}
 \newtheorem{Example}[ex]{Example}
 \newtheorem{ExamplesandDefinitions}[ex]{Examples and Definitions}
 \newtheorem{Remark}[re]{Remark}
 \newtheorem{Definition}[thm]{Definition}
%% For proofs, please use the proof environment (the amsthm package is loaded by the MDPI class).

%=================================================================
% Full title of the paper (Capitalized)
\Title{Transition-Metal Dopants in TiO$_2$ for Photocatalytic Nitrogen Fixation}

% Authors, for the paper (add full first names)

\Author{}
% Authors, for metadata in PDF

\AuthorNames{}

% Affiliations / Addresses (Add [1] after \address if there is only one affiliation.)
\address{
$^{1}$ \quad Affiliation 1; e-mail@e-mail.com\\
$^{2}$ \quad Affiliation 2; e-mail@e-mail.com}

% Contact information of the corresponding author
\corres{Correspondence: ajm@gatech.edu; Tel.: +x-xxx-xxx-xxxx}

% Current address and/or shared authorship
\firstnote{Current address: Affiliation 3} 
\secondnote{These authors contributed equally to this work.}

% Simple summary
%\simplesumm{}

% Abstract (Do not use inserted blank lines, i.e. \\) 


% Keywords
\keyword{keyword 1; keyword 2; keyword 3. List three to ten pertinent keywords specific to the article, yet reasonably common within the subject discipline.}

% The fields PACS, MSC, and JEL may be left empty or commented out if not applicable
%\PACS{J0101}
%\MSC{}
%\JEL{}

% If this is an expanded version of a conference paper, please cite it here: enter the full citation of your conference paper, and add $^\S$ in the end of the title of this article.
%\conference{}

%%%%%%%%%%%%%%%%%%%%%%%%%%%%%%%%%%%%%%%%%%
% Only for the journal Data:

%\dataset{DOI number or link to the deposited data set in cases where the data set is published or set to be published separately. If the data set is submitted and will be published as a supplement to this paper in the journal Data, this field will be filled by the editors of the journal. In this case, please make sure to submit the data set as a supplement when entering your manuscript into our manuscript editorial system.}

%\datasetlicense{license under which the data set is made available (CC0, CC-BY, CC-BY-SA, CC-BY-NC, etc.)}

%%%%%%%%%%%%%%%%%%%%%%%%%%%%%%%%%%%%%%%%%%
\begin{document}


\maketitle
\begin{abstract}

Developing novel, carbon neutral methods of generating fixed nitrogen is a key part of decarbonizing the global economy. TiO$_2$ has been observed to fix nitrogen at ambient pressure under illumination using only humidity and air. However, rates for this reaction remain low almost a half century after the process's discovery. In this work we investigate the use of surface metal dopants to promote surface reactions by altering the thermodynamics of intermediates on the surface. We screen all d-block transition metals for their ability to promote the nitrogen reduction reaction. We find that the formation energy of surface metal dopant sites follows a trend described by the d band model, with sites becoming more stable from left to right on the periodic table. The binding energies of N$_2$, N$_2$H, and NH$_2$ all reach maximums on metal sites in the middle of the d-block. These energies trend with the cohesive energy as calculated by the Friedel Model. The metal dopant systems show a volcano type relationship with respect to the NH$_2$ binding energy. The metals predicted to be most active are Ti, Rh, Tc, and Co, but these are not predicted to produce significant increases in rate.

\end{abstract}
\section{Introduction}
%%boilerplate Haber-Bosch intro
The fixation of atmospheric nitrogen has long been one of the prime challenges in chemistry and chemical engineering.\cite{ritter_18, Schloegl_2003} The Haber-Bosch process has been the route of choice for performing the fixation for the past century, permitting the most of the population growth over that period.\cite{Smil_1999} However, this process has many significant drawbacks, including high CO$_2$ emissions and heavy centralization requirements.\cite{Comer_2019} The Haber-Bosch process' considerable contribution to CO$_2$ emissions has been an increasingly pressing concern for the global community, as it is accountable 340 million tonnes of CO$_2$---fully 2\% of the carbon emissions worldwide.\cite{gross_12, Schiffer_2017} For this reason, supplanting the Haber-Bosch process could be a large contribution in global efforts to curb climate change. Another drawback lies in the necessity of centralization for Haber-Bosch, which contributes to global economic inequality.\cite{Comer_2019} Due to the high temperatures and pressures required, Haber-Bosch has significant economies of scale, meaning significant resources as well as a critical mass of demand is required for a new plant. Both of these conditions are met in industrialized nations, as the strong availability of capital allows for the necessary resources to be obtained and the agricultural industry provides the demand.\cite{McArthur_2017} However, these barriers have prevented developing regions, such as Sub-Saharan Africa, from developing Haber-Bosch plants, resulting in high costs of fertilizer for those regions, leading to reduced crop yields.\cite{yuan_2014, IFDC_2012} This causes fertilizer to be more expensive in poorer regions than in wealthy ones.

Due to the various drawbacks of the Haber-Bosch process, researchers have sought alternative means of producing fixed nitrogen \cite{Comer_2019, McPherson_2019,WANG20181055, Kyriakou_2017}. Two strategies seeing being examined are electrocatalysis\cite{McPherson_2019} and photocatalysis\cite{Medford_2017}. However, making either of these technologies viable presents a significant challenge. Electrochemical nitrogen fixation requires first generating electricity before conveying it to the catalyst surface to perform the reaction.\cite{kyriakou_2017} This extra step limits the ability of electrochemical processes to be decentralized, as power generation is needed in addition to the electrocatalytic cells. Another route has been photochemical nitrogen fixation, which provides a solution that only relies on sunlight and humidified air to produce fixed nitrogen. The photochemical route involves less capital investment and less local infrastructure because of its simplicity relative to electrochemistry, and is thus preverable for low resource environments.

%review photocatalytic nitrogen fixation experiments, then a paragraph on metal/nonmetal doping
Photochemical nitrogen fixation has been known to the scientific community for some time, but inconsistent results and low rates have discouraged further study.\cite{Medford_2017} While some would attribute the first discovery of photocatalytic nitrogen fixation to Dhar,\cite{Dhar_1941} the first well-controlled experiments were performed by Schrauzer and Guth.\cite{Schrauzer_1977} Schrauzer and Guth were able to establish the production of NH$_3$ in sterilized desert sands under illumination. These initial results spurred some skepticism, leading to follow-up experiments which included more precise markers, such as isotropic labeling.\cite{Schrauzer_1983} Many further experiments have been performed over the years attempting to establish that the production of NH$_3$ was not merely due to contamination\cite{Bickley_1979,Augugliaro_1982,Soria_1991,Li_2018,Yuan_2013,Hirakawa_2017} though the topic remains somewhat controversial. However, recent experiments utilizing ambient pressure X-ray Photoelectron Spectroscopy (AP-XPS) have observed reduced nitrogen compounds under only under illunation, strongly supporting the validity of previous experiments.\cite{Comer_2018b} Indeed, it has proven critical to ensure great care is taken in NH$_3$ measurement methods, due to the low concentrations and interference of other species.\cite{Gao_2018,Cui2018}

%describe difference between photochemical and electrochemical systems

%discuss efficiency
Rates of reaction for photocatalytic nitrogen fixation remain relatively low ($\mu mol$ scale \cite{Hirakawa_2017}), with no great improvements since the process's discovery. However, while it is often tacitly assumed that high efficiency is required to make a viable technology, it has been posited that only a comparatively low 0.1\% solar efficiency is required for feasibility.\cite{Comer_2019,Medford_2017} With sufficiently low capital cost, this system could see use in areas that are relatively far from fertilizer plants due to lowering of transportation costs. The competition of NH$_3$ production with H$_2$ evolution is the primary challenge for a potential electrocatalyst or photocatalyst, dubbed the "selectivity challenge".\cite{Singh_2017} For this reason, high faradic efficiency is also often sought in the electrochemical literature.\cite{McPherson_2019} The driver of this is the opportunity cost of using electricity for catalysis over other possible uses. Attention being focused on faradic efficiency often leads to the use of low overpotentials to obtain efficiency at the cost of reaction rate. Conversely, because the photoexcited electrons within a photocatalyst have no alternative uses there is no faradic efficiency requirement for a photocatalyst. A photochemical system could waste arbitrary amounts of it's catalytic turnovers producing H$_2$ gas, but be successful by achieving the 0.1\% solar to ammonia efficiency and reasonable reaction rates. %%write more on this

One one possible way of improving surface reaction rates is the inclusion of doping metals into TiO$_2$ samples. Doping metals can improve reaction rates by either altering the kinetics of the surface reaction or improving the material's photochemical properties of the material, such means as decreasing the band gap and tuning band alignment. Adding dopants is a common strategy to improve performance in TiO$_2$ photochemistry.\cite{Schneider_2014, Li_2007, Dozzi_2013} Schrauzer and Guth tested a variety of doping metals in their 1983 paper.\cite{Schrauzer_1983} In particular, the presence of iron was shown to improve NH$_3$ yields\cite{Schrauzer_1977,Augugliaro_1982}. However, the mechanism of iron appears to be based on charge separation rather than affecting the surface kinetics.\cite{Comer_2018} Hirakawa et al. found that depositing noble metals (Pt, Ir, etc.) onto an already prepared rutile (110) surface led to a decrease in reaction rates.\cite{Hirakawa_2017} They hypothesized that this was due to the metal atoms filling oxygen vacancy sites responsible for the reaction, though the vacancies they implicated are theoretically inactive.\cite{Comer_2018} 

While much is known about the ability of dopant metals to change the bulk properties of a material, little work has been done on how metal sites at the surface affect the surface kinetics. While there is a sizable literature on the effects of metal sites in 2D materials\cite{Khan_2018}, and single atoms on oxide supports\cite{Liu_2016} no such literature exists for metals sites embedded in oxide supports. In this work we focus on the ability of metals to improve surface kinetics, while setting aside their possible role in photophysics.

%review theoretical work

%To expedite the process, avoiding a multitude of experimental tests on the domain of possible dopants, there have been attempts to predict catalytic activity computationally. Computational screening has been used profitably in several areas of catalysis including CO$_2$ reduction\cite{}, hydrogen evolution\cite{}, and oxygen reduction\cite{Norskov_004}. These theoretical studies screening materials for effectiveness in nitrogen fixation have confirmed its applicability as a predictive method.\cite{Hoskuldsson_2017} Theoretical results verify that the N$_2$ bond dissociation is the rate-limiting step. \cite{https://pubs.acs.org/doi/pdf/10.1021/jp056982h} Further endeavors focus on determining specific reaction pathways or finding more efficient ceramics, surfaces, and dopants.\cite{}
%describe challenges of ambient temperature N2 fixation

The two largest challenges to fixing nitrogen under ambient conditions have been posited to be the kinetics of the first hydrogenation step and the desorption of NH$_3$.\cite{Hoskuldsson_2017,Singh_2017,Montoya_2015} These two challenges provide the basis of two design criteria for nitrogen fixation catalysts. N$_2$ is a molecules with no intrinsic dipole and a very strong N-N triple bond making reactions challenging at room temperature.\cite{Montoya_2015,Comer_2018}. For this reason, N$_2$ is often referred to as being inert. This is the primary reason the first hydrogenation step is found to be rate limiting for low temperature nitrogen reduction.\cite{Montoya_2015, Singh_2017, Hoskuldsson_2017, Comer_2018} The exceedingly high temperatures of the Haber-Bosch process (700K) allow it to follow a "dissociative" mechanism, whereby N$_2$ is dissociated on the catalyst surface into nitrogen adatoms immediately in the first step.\cite{Ullmann_amm_2006, Hellman_2006} Surmounting such a high energy barrier is not feasible at room temperature, requiring the reaction to follow an "associative" mechanism in which hydrogen atoms are attached successively to the nitrogen until the N-N bond can be broken.\cite{Montoya_2015}

% review what we did in this publication then lead into results
In this work, we examine the potential of doping metal atoms to improve the surface chemistry of nitrogen fixation on the rutile (110) surface. We screen The d-block transition metals substituted onto the (110) surface with an added vacancy at the bridging oxygen (see Figure \ref{}) rather than lying on the surface as in single atom catalysis. The resulting metals hold the 2+ oxidation state. We demonstrate the trends present across the periodic table with the formation energy, N$_2$ adsorption energy, and N$_2$H adsorption energy. We also map out the thermodynamics of both associative and dissociative pathways on these materials to explore their suitability toward N$_2$ reduction reactions. Finally, we assess the expected improvement in reaction rates that is expected from forming metal dopant sites on the surface for photocatalytic reactions. We find that none of the examined candidates improve the reaction rate significantly. This finding agrees with findings in the experimental literature, that the addition of metals does not improve reaction rates for photocatalytic nitrogen fixation.

\section{Results and Discussion}
Rutile (110) was chosen as a model surface due to rutile's implication in increased rates of reaction.\cite{Schrauzer_1977} Addtionally, there is a rich literature on the surface science of rutile (110)\cite{Diebold2003,Yates_1991,Lu1994,Walle2009}. From this model surface slabs containing metal dopants at the surface in the 2+ state were generated for each studied dopant metal. These 2+ sites were created by replacing the 6 fold titanium site with the substituent metal and removing a bridging oxygen (see Figure \ref{fig:ex_slab}). The 2+ oxidation state was chosen, as all transition metals are able to hold a 2+ oxidation state\cite{Greenwood_chemistry_text_book}, allowing for trends to be seen. The lattice parameters of the unit cells were fixed at the calculated value for pure rutile TiO$_2$. Spin polarization was implemented in all simulations to ensure the lowest energy spin state was obtained for each site. In total, 24 surfaces were screened for their surface formation properties and potential to perform nitrogen fixation by mapping out the thermodynamics of all possible nitrogen reduction pathways on the surface. Full details of the calculation methodology can be found in the methods section.

\begin{figure}
    \centering
    \includegraphics[width=0.3\linewidth]{Images/ex_slab.png}
    \caption{An example of the screened slabs. The substituent metal has replaced a 6 fold Ti atom (seen in green) and a bridging oxygen vacancy has been formed to allow the metal to enter the 2+ oxidation state.}
    \label{fig:ex_slab}
\end{figure}

\subsection{Trends Across The Periodic Table}

\subsubsection{Active Site Formation Energies}

Studying these metal substituted active sites has revealed trends in the formation energies of these active sites across the periodic table. These trends appear in both the formation energies of these sites and the binding energies of species on the sites. The nature of these trends is based in electronic interactions between the metals' d-band elections and the surface and adsorbates. The dependance of catalytic properties on the properties of a metal's d-band is a concept commonly seen transition metal catalysis\cite{Hammer_2000} with the most prominent example being the d-band model\cite{Nilsson_2008, Greeley_2002}. However, full analysis of the electronic structures of these active sites is beyond the scope of this work, which focuses on the practical implications of the trends on catalytic systems.

The stability of substituted metal surface sites has been examined with respect to the position of their d band center. The formation energy if each metal substitute was calculated with respect to their pure metallic form and plotted in Figure \ref{fig:d_band}. The plot indicates there is a strong correlation between these two quantities, showing that the ability for a metal to form 2+ surface sites is strongly related to the position of its d-band center. We can rationalize the observed correlation in the context of the d band model of chemical bonding\cite{Nilsson_2008} summarized in Equation \ref{fig:d_band} below:

\begin{equation}
    %E_{coh} = \epsilon_d + \epsilon_s \\
    \Delta E_d = \int^{E_F} E(\rho'(E) - \rho(E))dE
    \label{eq:d_band}
\end{equation}

where $\Delta E_d$ is the energy of binding associated with interaction with the d band, $E_F$ is the fermi level energy, $\rho(E)$ is the d band density of states of the material, and $\rho'$ is the density of states after adsorption. Within the framework of this model, the interaction between s band states and the adsorbate is approximately the same for all metals, but the interaction with the d band varies from metal to metal. For the case of adsorbed species as the d band center's energy moves closer to the fermi level from below more anti-bonding energy states are being filled, weakening the bond.

In this case the system involves a metal integrating into an oxide surface rather than an adsorbate binding to the surface. The comparison applies because the d bands of the integrated metal atom are mixing with the bands of the oxygen atoms in the surface. Thus, as the d band model predicts the interaction weakens from left to right on the periodic table. This implies that the metals most able to integrate into a surface are those with the most favorable interaction with oxygen. A similar relationship is seen with the oxide formation energy of metals and their tendency to show strong metal support interactions\cite{O_Connor_2018}, with more favorable oxide formation energies showing stronger binding. This work suggests that the d band center may also be a helpful descriptor for metal support interactions.

The only exceptions to this trend appear to be Ti, Zr, Hf, and Ag. The first three can be rationalized fairly easily, as all three lie in the same column of the periodic table which is the same column as the host metal, Ti. This affords approximately 1.5eV of improved stability relative to the trend. This correlation lines up with the chemical intuition that elements in the same column are most able to substitute one another chemically. The final outlier, Ag, is more difficult to explain. When the elements are plotted with respect to their column on the periodic table (see Figure SXX), Ag is no longer an outlier, but periodic column is overall a less powerful descriptor explanatory descriptor (R$^2$ = 0.87). However, the outlier status of Ag may indicate that the analysis presented is not complete.

These data also have implications for the formation of surface metal clusters. Figure \ref{fig:d_band} indicates that when considering a dopant metal's tendency to form surface segregated sites or single atom sites rather than integrate into the surface on TiO$_2$, considering the element's d band center is critical. This also has implications for the ability to synthesize metal doped surfaces experimentally, showing that some elements (Y, Sc, Zr, Hf) favor integration into the surface structure rather than the formation of surface metal clusters. Conversely, traditional catalysis metals such as Rhodium and Platinum do not integrate into the surface favorably and will tend to form surface nano-clusters, even under thermodynamically equilibrated conditions. This result agrees with TEM measurements in the experimental literature, indicating that clusters of metals such as platinum and silver form on TiO2 surfaces.\cite{Iliev_2006}\todo{find some more TEM work}  While this work only examines TiO$_2$, similar trends may exist in other metal oxide materials, warranting further investigation. It should also be emphasized that this result refers to the formation of a 2+ surface site, not integrating into the bulk structure.

\begin{figure}
    \centering
    \includegraphics[width=0.8\linewidth]{Images/d_band_vs_formation.pdf}
    \caption{Surface site formation energy of surface sites with respect to their bulk metalic form vs the substituted metal's d band center as reported by Ruban et. al \cite{Ruban_1997}}
    \label{fig:d_band}
\end{figure}

%In Figure \ref{fig:2+_N2_react_stab} the formation energy of the surface vs binding energy of N$_2$ are shown. Both of these values are important to quantify, as an active site with an exceptionally high formation energy is unlikely to exist on the surface in any significant quantity. Because of the close relationship between the formation energy of 2+ metal substitutes and the valence number seen in Figure \ref{fig:valence} this plot may also be read as approximately showing the N$_2$ adsorption energy across the periodic row similar to Figure \ref{fig:2+_N2_period}. From a There are many sites that bind N$_2$ with an energy of $\approx$ -0.2eV, which represents a weak physorption interaction and is not sufficient to bind N$_2$ at room temperature. However, some metal sites bind N$_2$ in a manner related to the site's formation energy. The sites able to bind N$_2$ are all in the middle of the row as seen in Figure \ref{fig:2+_N2_period} This fact highlights the tradeoff between two trends: the adsorption of N$_2$ and surface formation energy. These metal sites yield a frontier of Pareto optimal surface sites for adsorbing N$_2$. This analysis indicates that the active site's relative instability is a necessary but not sufficient condition for increasing N$_2$ binding.

\subsubsection{Binding of Nitrogen Species}

Patterns in N$_2$ and N$_2$H can also be seen down the rows of the periodic table as seen in Figures \ref{fig:N2_rows}. It should be noted that the results for group 4 have not been included in the main text due to difficulties converging the electronic structures of the materials, making trends difficult to infer. The results for elements in group 4 that did converge may be seen in the supplementary information. The likely reason is the challenging magnetic states of several systems within this row. Figure \ref{fig:2+_N2_period} particularly shows the trend of increased N$_2$ binding down the periodic row, reaching a maximum in the center of the row and dropping more rapidly as the d-shell is filled. The results do not match the traditional d-band model\cite{Nilsson_2008} for transition metal binding energies which would have the adsorption energy getting weaker down the row. This is not surprising as the d band model was derived for transition metals, not metal oxides. In addition to the trend across the rows, the binding strength increases down the columns with Os having the strongest binding as well as being in the middle of the d-block and on the lowest tested row for N$_2$ binding. Finally, it can also be noted that the position of the maxima for N$_2$ adsorption within a rows is not consistent, with row 6 having a maxima in the middle of the period, whereas row 5 has a maxima one position to the right.

\begin{figure}

\begin{subfigure}
    
    \centering
    \includegraphics[width=0.5\textwidth]{Images/N2_adsorption_rows.pdf}
    \caption{Caption}
    \label{fig:N2_rows}
\end{subfigure}

\begin{subfigure}
    \centering
    \includegraphics[width=0.5\textwidth]{Images/N2H_adsorption_rows.pdf}
    \caption{Caption}
    \label{fig:N2H_rows}
\end{subfigure}
\end{figure}



While the trend in N$_2$ and N$_2$H binding observed does not match the traditional d band model explanation the binding energies of these species does correlate with the d band contributions of the cohesive energies of the corresponding bulk systems (Figure \ref{fig:N2H_cohesive}). The d band cohesive energy can be described by the The Friedel Model \cite{1969TPom} from solid state physics summarized in Equations \ref{eq:cohesive} and \ref{eq:d_band_cohesive}:


\begin{equation}
    E_{coh} = \epsilon_d + \epsilon_s
    \label{eq:cohesive}
\end{equation}

\begin{equation}
    \epsilon_d = \int^{E_F} (E_d-E)\rho(E)dE
    \label{eq:d_band_cohesive}
\end{equation}

where $E_{coh}$ is the cohesive energy, $\epsilon_d$ is the d contribution to the cohesive energy, $\epsilon_s$ is the s contribution to the cohesive energy, $E_F$ is the fermi level energy, $E_d$ is the energy of the d band broadening, $\rho(E)$ is the d band density of states of the material. 

The similarity between the Friedel Model (Equations \ref{eq:cohesive} and \ref{eq:d_band_cohesive}) and the d band model (Equation \ref{eq:d_band}) should be noted. Both models involve the filling of bonding and anti-bonding d orbitals. However, in the Friedel Model, the energy change is from fully isolated metal atoms to crystalline transition metals. A result of the Friedel Model is that in the first half of the transition metal row, the bonding orbitals are able to fill first followed by the anit-bonding orbitals in the second half of the row. The result is that cohesive interactions strengthen in in the first half of the row and weaken in the second half, creating a maxima near when the d shells are approximately half filled. The likely reason is that both phenomena involve populating the d band of metals, and depend on the total energy of filling the metal's d band up to the fermi level (Equation \ref{eq:d_band_cohesive}). This suggests that the filling of the metal's d band for binding nitrogen species first fills up d bonding orbitals moving down the periodic row rather than merely anti-bonding orbitals as in the case of the d-band model.

%when doped on a surface and the metal sublimating are similar.

%Thus, we hypothesize that these phenomena are governed by the same underlying physics. 

A similar trend is seen with N$_2$ adsorption in Figure \ref{fig:N2_cohesive}. however the correlation is much less predictive due to several elements displaying much stronger binding than would be expected by this model. The low $R^2$ value of this correlation implies that the system is more complicated than the case of N$_2$H. Thus, for the case of N$_2$ binding, increasing cohesive energy predicting increased binding is only a general trend. 

The existence of these trends shows the potential for tuning embedded metal sites to perform particular reactions. While work has been done showing the effect of changing the structure of the metal oxide\cite{Back_2019, Hoskuldsson_2017}, it may prove useful to modify the identity of single sites of the surface to improve reactivity. We hypothesize that trends similar those found on TiO$_2$ apply to other metal oxides and crystal structures and present the d band contribution to the cohesive energy as a descriptor for this process.

%In the  shows a reasonable correlation with the d band energy contribution to the cohesive energies of the bulk phases of the substituent metals as seen in Figure \ref{fig:N2H_cohesive}. This quantity may be obtained using equation \ref{eq:d_band_cohesive} below \cite{Gautier_1975}. The relationship implies that the d band contribution of the cohesive energy is related to the reactivity of the metal atom. Larger cohesive energy implies that the metal is more stabilized through bonding with itself, which can be thought of as the metal being more reactive. This trend does not match the one typically seen in the d band model, however the relationsh More reactive elements tend to lie in the center of the periodic row, due to their d band filling properties. The Friedel Model\cite{1969TPom} describes this in terms 




\begin{figure}
    \centering
    \includegraphics[width=0.5\textwidth]{Images/cohesive_eng_vs_N2H.pdf}
    
    \caption{The d band contribution to the cohesive energy of transition metals vs the N$_2$H binding energy of the metal substituted in the 2+ oxidation state in TiO$_2$. d band cohesive energy contributions obtained from Turchanin and Agraval\cite{Turchanin_2008}}
    \label{fig:N2H_cohesive}
\end{figure}

\begin{figure}
    \centering
    \includegraphics[width=0.5\textwidth]{Images/cohesive_eng_vs_N2.pdf}
    
    \caption{The d band contribution to the cohesive energy of transition metals vs the N$_2$ binding energy of the metal substituted in the 2+ oxidation state in TiO$_2$. d band cohesive energy contributions obtained from Turchanin and Agraval\cite{Turchanin_2008}}
    \label{fig:N2_cohesive}
\end{figure}

\begin{figure}
    \centering
    \includegraphics[width=0.5\textwidth]{Images/cohesive_eng_vs_NH2.pdf}
    
    \caption{The d band contribution to the cohesive energy of transition metals vs the N$_2$ binding energy of the metal substituted in the 2+ oxidation state in TiO$_2$. d band cohesive energy contributions obtained from Turchanin and Agraval\cite{Turchanin_2008}}
    \label{fig:N2_cohesive}
\end{figure}

%\begin{figure}
%    \centering
%    \includegraphics[width=0.8\linewidth]{Images/4+N2_Formation.pdf}
%    \caption{4+ Formation Energy versus N2 Adsorption Energy}
%    \label{fig:4+_N2_react_stab}
%\end{figure}
%\begin{figure}
%    \centering
%    \includegraphics[width=0.8\linewidth]{Images/2+N2_Formation.pdf}
%    \caption{2+ Formation Energy versus N2 Adsorption Energy}
%    \label{fig:2+_N2_react_stab}
%\end{figure}
%The first hydrogenation step is commonly thought to be rate limiting, thus the third criterion is a reasonable N$_2$ $\rightarrow$ N$_2$H reaction energy. While some would also add NH$_2$ binding energy as a fourth criterion\cite{}, this was not evaluated. \todo{give a better reason why we didn't do this} However, these authors suggest that an N$_2$H binding energy of 0eV is the optimum for rutile oxides. 


%Based on this analysis we have identified X metals which show promise for photocatalytic nitrogen fixation. We completed reaction pathways for each of these reactions, the results can be seen in Figure \ref{}.\todo{discuss the results once we have the data}
%The adsorption of N$_2$ on dopant metals follows a clear trend across the rows of the periodic table as well as down the periodic columns (Figure \ref{}). The trend can be generally described by the d band model.\cite{Nilsson_2008}\todo{describe the d band model} 
%D-band theory is an explanation of one factor involved in deciding the strength of the interaction between the molecule and the adsorbate. Because of the similarity of the s and p blocks of the transition metals, the most important interactions between the metal adsorbate and the molecule are with the d band. When the molecule and the metal adsorbate interact, they create both a bonding and antibonding state. The higher the filling of the antibonding state, the weaker the bonding is. This corresponds inversely with the height of the d band center; the higher the d band center, the less the antibonding state is filled. This theory predicts that near the center of the d block, the interactions will be stronger, corroborating our theory.




%\begin{figure}
%    \centering
%    \includegraphics[width=0.8\linewidth]{Images/2+N2_by_element.pdf}
%    \caption{2+ N2 Adsorption Energy versus Element}
%    \label{fig:2+_N2_period}
%\end{figure}

%\begin{figure}
%    \centering
%    \includegraphics[width=0.8\linewidth]{Images/2+N2H_by_element.pdf}
%    \caption{2+ N2H Adsorption Energy versus Element}
%    \label{fig:2+_N2H_period}
%\end{figure}
%\begin{figure}
%    \centering
%    \includegraphics[width=0.8\linewidth]{Images/2+N2H_Formation.pdf}
%    \caption{2+ Formation Energy versus N2H Adsorption Energy}
%    \label{fig:2+_N2H_react_stab}
%\end{figure}
%\begin{figure}
%    \centering
%    \includegraphics[width=0.8\linewidth]{Images/4+N2_by_element.pdf}
%    \caption{4+ N2 Adsorption Energy versus Element}
%    \label{fig:4+_N2_period}
%\end{figure}

%\begin{figure}
%    \centering
%    \includegraphics[width=0.8\linewidth]{Images/4+N2H_by_element.pdf}
%    \caption{4+ N2H Adsorption Energy versus Element}
%    \label{fig:4+_N2H_period}
%\end{figure}
%\begin{figure}
%    \centering
%    \includegraphics[width=0.8\linewidth]{Images/4+N2H_Formation.pdf}
%    \caption{4+ Formation Energy versus N2H Adsorption Energy}
%    \label{fig:2+_N2H_react_stab}
%\end{figure}

%There is an inherent uncertainty in predicting energies of reactions using DFT. Literature suggests an additional error buffer of around 0.2 eV (CITATIONS). This does not greatly change the interpretation of our results, as the margin of favorability or disfavorability for the calculated reactions is typically larger than 0.2 eV. The change in the likelihood of a reaction occurring with the energetic favorability becomes smaller with increasing magnitude of energies, due to the exponential distribution of temperature.

\subsection{Reactivity Toward Nitrogen Fixation}

To probe the ability of the generated active sites to perform the N$_2$ reduction reaction, the full thermodynamics of the N$_2$ reduction reaction pathways of all elements were calculated. To simulate the effect of excited electrons, the computational hydrogen electrode was used and the potential was set to that of the band edge of rutile TiO$_2$. We therefore neglect the effects these metals could have on the photo-physical properties of the material. The free energy diagrams for each pathway considered are available in the supplementary information.

To examine the results of these free energy diagrams, the energy of the largest reaction barrier was plotted against the NH$_2$ binding energy. Figure \ref{fig:NH2_limiting} summarizes the results of this analysis. This plot reveals an approximate volcano type relationship between the NH$_2$ binding energy and the largest barrier of the rate limiting step. Unlike Hoskuldsson et. al.\cite{Hoskuldsson_2017} we find that the NH$_2$ binding energy to be the most powerful descriptor rather than N$_2$H binding.

One outcome of note in Figure \ref{fig:NH2_limiting} is the number of metals that lie near the top of the volcano. We find that Rh, Ti, Tc, and Co all display similar NH$_2$ binding and also similar values for the limiting step energy. We also see that the limiting step does not cleanly shift on each side of the volcano, as is normally seen in similar analysis, indicating the predictive power of scaling relations of the type identified by N{\o}rskov et. al.\cite{N_rskov_2004}. The computed limiting step energies for the metals at the top of the volcano all lie within the error range of the GGA approximation with respect to one another (~0.2eV)\cite{}, thus strong conclusions cannot be made about. Nonetheless, We predict a small improvement in rates when generating Co 2+ active sites on the surface. However, because of the inherent uncertainty in DFT energies, the true binding energy of any of these four metals may be higher or lower by a few tens of millielectron volts, thus warranting experimental investigation.

Discouragingly, one of the species at the top of the volcano is the host metal of the oxide material, Ti. This means that substituting other metals is unlikely to have a significant effect on rates of reaction for photocatalytic reactions. Despite the limitations of metal surface metal doping for TiO$_2$, based on the ability of dopant sites to modify the thermodynamics on the surface we assert that the strategy of generating dopant sites is a fruitful path forward for improving the reactivity of oxides. The lack of improvement in the rate of reaction with metal addition also agrees with the previous experimental findings by Hirakawa et. al.\cite{Hirakawa_2017} on TiO$_2$ which show no improvements in reaction rates when metals are bound to the surface.


\begin{figure}
    \centering
    \includegraphics[width=0.5\textwidth]{Images/NH2_v_rate_limiting.pdf}
    
    \caption{The highest barrier observed vs the NH$_2$ Binding Energy}
    \label{fig:NH2_limiting}
\end{figure}

These results highlight the challenge of improving reaction rates for photocatalytic nitrogen fixation, indicating that more creative strategies must be employed. While have not considered the possible effects of metals on TiO$_2$'s bulk photochemical properties such as charge separation and carrier lifetime we believe this provides a starting point for engineering surface segregated metal dopants. In future, we could imagine engineering oxide structures with metal dopants in the bulk structure that improve photochemical properties and surface segregated metal dopants that improve reaction kinetics.

\section{Conclusions}
This work has investigated stability of metal dopant surface sites and their effects on the reaction thermodynamics of N$_2$ reduction on rutile (110). We find that the formation energy of these doped surface states is strongly related to the location of the d band center of the subsituted metal, with the trend matching the d band model. We hypothesize that a similar relationship may exist for integrating surface dopant metals in other oxide materials. We also find correlation between the cohesive energy of metals and their N$_2$H and NH$_2$ binding energy on the surface. Finally, we investigate the effects of dopant sites on the full reaction pathways for the studied metals. We find a volcano type relationship between NH$_2$ binding and the highest reaction barrier in the pathway for photocatalytic rations. However, we conclude that no metal substitute is able to significantly improve the thermodynamics of the surface due to the host metal, Ti, being near the top of the volcano. While this work does not demonstrate the capabilities of substituted metal sites to improve rates of reaction, we show a substantial ability to modify reaction thermodynamics on the surface. For this reason, generating metal dopant sites on the surface is a potential path forward for improving reaction rates on metal oxide surfaces.


\section{Methods}
\subsection{Density Functional Theory Calculations}
All first principles calculations were performed in the Quantum Espresso software package \cite{QE-2009}.
The TiO$_2$ slabs and atomistic images were created using the Atomic Simulation Environment (ASE) package\cite{Hjorth_Larsen_2017}. It is a 110 surface, two Ti atoms long and wide and four Ti atoms deep. The pristine slab totals 48 atoms, with 4 Ti and 8 O per layer. Either of the two non-identical types of Ti atoms on the top layer were changed to a different transition metal to make 2+ or 4+ oxidation states. One bridging O was removed to create the vacancy for the added gas to adsorb at. 6 Angstroms of vacuum were added on both top and bottom, to avoid adverse effects of periodicity in the z direction.

For the DFT calculations, the bottom 4 Ti and 14 O atoms were fixed in place to aid convergence. Periodic boundary conditions were implemented in all three dimensions, but the vacuum in the z direction separated the surface of interest. A plane wave cutoff of 400 eV and a Monkhorst-Pack k-point grid with a spacing of 4 by 4 by 1 was used \cite{Monkhorst_1976}. Dipole corrections and spin polarization were included, and the magnetic moments were perturbed for each calculation. The convergence threshold was selected as $10^{-6}$ eV, and BEEF-vdW functionals \cite{Wellendorff_2012} with Fermi-Dirac smearing were used.

The adsorption energies were obtained from the DFT calculations by subtracting the energy of a clean slab and the energy of the free gas molecule from the energy of the gas adsorbed to the slab.
\begin{equation}
E_{adsorption} = E_{slab+adsorbate} - E_{slab} - E_{adsorbate}
\end{equation}
The error of DFT calculations tends to be in the order of magnitude of 0.5 eV \cite{Gautier_2015}. The BEEF-vdW functional ensemble allows an estimate of the error to be made. The calculation of the energy for a given reaction was repeated for each functional individually, resulting in a distribution of energies. Error bars were determined from this by finding the standard deviation of 

\subsection{Thermochemistry}
\cite{ase-paper,Reuter_2005}

To calculate the adsorption energy at standard temperature and pressure, the Thermochemistry package from ASE was used \cite{ase-paper}. Free gasses were approximated in the ideal gas limit, and adsorbed gasses in the harmonic limit. The ideal gas limit thermochemistry module allows translation and rotation in all directions, and assumes these modes are independent. The harmonic limit thermochemistry module approximates vibrations in all directions as harmonic oscillators. The two approximations give the Gibbs and Helmholtz free energies, respectively. The Helmholtz free energy is equivalent to Gibbs, if the product of the pressure and the volume is taken to be negligible, which was assumed in this case. From these results, the free energies were calculated using \begin{equation}
    G=H-TS
\end{equation}. The same DFT parameters were used for calculating the free energies of adsorption as for the non-entropic results.

\subsection{Photochemistry}
The photochemistry has been treated using the methods outlined by Hellman et. al. \cite{Hellman2017}. Within this framework, the effects of excited states are neglected allowing the treatment of excited electrons and holes using the computational hydrogen electrode model (CHE). The potentials of electrons and holes are set at the value of the band edges of the semiconductor.




\bibliographystyle{mdpi}
\bibliography{main}
\end{document}
